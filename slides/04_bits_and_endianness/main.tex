\documentclass[hyphens,aspectratio=169,dvipsnames]{beamer}
\usepackage{graphicx}
\usepackage{xcolor}
\usepackage{amssymb}
\usepackage{amsmath}
\usepackage{mathtools}
\usepackage{textcomp}
\usepackage{moresize}
\usepackage{framed}
\usepackage{minted}
\usepackage{relsize}
\usepackage{tikz}
\usepackage{comment}
\usetikzlibrary{shapes.geometric, arrows, positioning}
\usetheme{Berlin}
\usecolortheme[RGB={0,95,47}]{structure}
\beamertemplatenavigationsymbolsempty

\makeatletter
\AtBeginEnvironment{minted}{\dontdofcolorbox}
\def\dontdofcolorbox{\renewcommand\fcolorbox[4][]{##4}}
\makeatother

\newcommand{\textpf}[1]{\texttt{\color{black}\fcolorbox{lightgray}{lightgray}{#1}}}

\begin{document}

\section{Binary and Hex}

\begin{frame}{Binary and Hexadecimal}
    \begin{itemize}
        \pause \item Why do we use binary?
        \pause \item Why do we use hex?
    \end{itemize}
    % Whiteboard as necessary here.
\end{frame}

\section{2's Complement}

\begin{frame}{Negative Binary Numbers}
    How should we represent negative numbers in binary?
\end{frame}

\begin{frame}{Take 1: Sign Bit}
    \begin{itemize}
        \pause \item Reserve the most significant bit as the sign bit.
        \pause \item To negate a number, flip its sign bit.
    \end{itemize}

    \begin{align*}
        \texttt{bin}(-25) &= \texttt{set\_msb}(\texttt{bin}(25)) \\
                          &= \texttt{set\_msb}(\texttt{0b}00011001) \\
                          &= \texttt{0b}10011001
    \end{align*}
\end{frame}

\begin{frame}{Sign Bit Pros and Cons}
    Pros
    \begin{itemize}
        \pause \item It's easy for people to understand.
    \end{itemize}

    \pause

    Cons
    \begin{itemize}
        \pause \item +0 and -0 are distinct.
        \pause \item Calculations are not consistent with unsigned numbers.
    \end{itemize}

    \pause
    \begin{center}Whiteboard!\end{center}
\end{frame}

\begin{frame}{Take 2: 2's Complement}
    \begin{itemize}
        \pause \item To negate a number, flip all of its bits, then add 1.
        \pause \item This is how nearly all modern computers represent negative numbers.
    \end{itemize}

    \pause \begin{align*}
        \texttt{bin}(-25) &= \texttt{flip\_bits}(\texttt{bin}(25)) + \texttt{0b}1 \\
                          &= \texttt{flip\_bits}(\texttt{0b}00011001) + \texttt{0b}1 \\
                          &= \texttt{0b}11100110 + \texttt{0b}1 \\
                          &= \texttt{0b}11100111
    \end{align*}
\end{frame}

\begin{frame}{2's Complement Pros and Cons}
    Pros
    \begin{itemize}
        \pause \item There's only one zero.
        \pause \item Calculations are consistent with unsigned numbers!
    \end{itemize}

    \pause
    \begin{center}Whiteboard!\end{center} % -18 -38
    \pause

    Cons
    \begin{itemize}
        \pause \item It's more difficult to understand.
    \end{itemize}
\end{frame}

\begin{frame}[fragile]{Example Program}
    What does this program do?
    \begin{center}
        \begin{minted}{c}
#include <stdint.h>
#include <stdio.h>

int main(void) {
    int8_t i8 = -1;
    int16_t i16 = i8;
    printf("%hd\n", i16); // "%hd" means base-10 signed short
}
        \end{minted}
    \end{center}
\end{frame}

\begin{frame}{Sign Extension}
    \begin{itemize}
        \pause \item When converting a negative integer from a smaller bitwidth to a larger bitwidth, you need to fill the new bits with \texttt{1}s in order to preserve the original integer's value.
    \end{itemize}

    \pause For example,
    \begin{align*}
        \texttt{(int16\_t)(int8\_t)(-1)} &= \texttt{(int16\_t)0b11111111} \\
                                         &= \texttt{0b}\underbrace{\texttt{11111111}}_\text{(8 new \texttt{1}s)}\texttt{11111111} \\
                                         &= \texttt{(int16\_t)(-1)}
    \end{align*}
\end{frame}

\begin{frame}{Sign Extension}
    \begin{itemize}
        \pause \item When converting a nonnegative integer from a smaller bitwidth to a larger bitwidth, you need to fill the new bits with \texttt{0}s in order to preserve the original integer's value.
    \end{itemize}

    \pause For example,
    \begin{align*}
        \texttt{(int16\_t)(int8\_t)(127)} &= \texttt{(int16\_t)0b01111111} \\
                                          &= \texttt{0b}\underbrace{\texttt{00000000}}_\text{(8 new \texttt{0}s)}\texttt{01111111} \\
                                          &= \texttt{(int16\_t)(127)}
    \end{align*}
\end{frame}

\section{Endianness}

\begin{frame}[fragile]{Example 1}
    Q: What does this print?
    \begin{minted}{c}
#include <stdint.h>
#include <stdio.h>

int main(void) {
    uint8_t i[] = {0xFF, 0x00};
    uint16_t *p = i;
    printf("%x\n", *p);
}
    \end{minted}

    \pause A: Either \texttt{0xFF00} or \texttt{0x00FF}, depending on the endianness of your CPU.
\end{frame}

\begin{frame}{Endianness}
    CPUs can be \textbf{little-endian} or \textbf{big-endian} (LSB or MSB, le/el or be). \\

    When a little-endian CPU executes a \textbf{multi-byte} read, it reverses the order of the bytes before processing them. \\

    When a little-endian CPU executes a \textbf{multi-byte} write, it writes the bytes from least-significant to most-significant.
\end{frame}

\begin{frame}{Examples of Little-Endian Reads}
    Suppose that a computer's memory had the following contents at address \texttt{p}:
    \begin{center}
        \begin{tabular}{c|c|c|c|c|c}
            \hline
            ... & \texttt{0xDE} & \texttt{0xAD} & 0\texttt{xBE} & \texttt{0xEF} & ... \\
            \hline
        \end{tabular}

            $\overset{\uparrow}{\texttt{p}}~~~~~~~~~~~~~~~~~~~~~~~~~~~~~~~~~~~~~~$ \\ % This is a hack :)

    \end{center}

    When I read 2 bytes from \texttt{p} on a big-endian machine, I get... \pause \texttt{0xDEAD}.

    When I read 2 bytes from \texttt{p} on a little-endian machine, I get... \pause \texttt{0xADDE}.

    When I read 4 bytes from \texttt{p} on a big-endian machine, I get... \pause \texttt{0xDEADBEEF}.

    When I read 4 bytes from \texttt{p} on a little-endian machine, I get... \pause \texttt{0xEFBEADDE}.
\end{frame}

\begin{frame}{Endianness Misconception 1}
    Endianness is about bytes, \textbf{not bits}.

    A little-endian system reverses the order of bytes during memory accesses.

    \begin{center}\textbf{It does not change the order of the bits within each byte.}\end{center}

    % As a consequence, if you always read and write things 1 byte at a time, you will not observe any differences between big- and little-endian systems.
\end{frame}

\begin{frame}[fragile]{Endianness Misconception 2}
    Endianness does not affect the direction of memory accesses.

    That is, if we execute a 4 byte read at the address \texttt{0x4000}, then we're reading from addresses \texttt{0x4000}, \texttt{0x4001}, \texttt{0x4002}, and \texttt{0x4003}, regardless of the endianness of the system.
\end{frame}

\begin{frame}[fragile]{Example 2}
    What does this print on a little-endian computer? On a big-endian computer?
    \begin{minted}{c}
#include <stdint.h>
#include <stdio.h>

int main(void) {
    uint32_t data = 0xDEADBEEF;
    uint8_t *p = (uint8_t *)&data;
    printf("%hhx %hhx %hhx %hhx\n", p[0], p[1], p[2], p[3]);
}
    \end{minted}
\end{frame}

\begin{frame}[fragile]{???}
    \begin{center}Why do we do this to ourselves?\end{center}

    \pause
    \begin{center}\textbf{Optimization.}\end{center}

    \begin{minted}{c}
#include <stdint.h>

uint8_t f(uint64_t *p64) {
    return *p64;
}
    \end{minted}
\end{frame}

\begin{frame}[fragile]{The Why}
    \begin{center}
        \begin{columns}
            \begin{column}{0.5\textwidth}
                Big-endian
                \begin{minted}{asm}
f:
        addiu   $sp, $sp, -8
        sw      $ra, 4($sp)
        sw      $fp, 0($sp)
        move    $fp, $sp
        lbu     $2, 7($4)
        move    $sp, $fp
        lw      $fp, 0($sp)
        lw      $ra, 4($sp)
        jr      $ra
        addiu   $sp, $sp, 8
                \end{minted}
            \end{column}
            \begin{column}{0.5\textwidth}
                Little-endian
                \begin{minted}{asm}
f:
        addiu   $sp, $sp, -8
        sw      $ra, 4($sp)
        sw      $fp, 0($sp)
        move    $fp, $sp
        lbu     $2, 0($4)
        move    $sp, $fp
        lw      $fp, 0($sp)
        lw      $ra, 4($sp)
        jr      $ra
        addiu   $sp, $sp, 8
                \end{minted}
            \end{column}
        \end{columns}
    \end{center}
\end{frame}

\end{document}
