\documentclass[hyphens,aspectratio=169,dvipsnames]{beamer}
\usepackage{graphicx}
\usepackage{xcolor}
\usepackage{amssymb}
\usepackage{amsmath}
\usepackage{mathtools}
\usepackage{textcomp}
\usepackage{moresize}
\usepackage{framed}
\usepackage{minted}
\usepackage{relsize}
\usepackage{emoji}
\usepackage{tikz}
\usepackage{comment}
\usetikzlibrary{shapes.geometric, arrows, positioning}
\usetheme{Berlin}
\usecolortheme[RGB={0,95,47}]{structure}
\beamertemplatenavigationsymbolsempty

\makeatletter
\AtBeginEnvironment{minted}{\dontdofcolorbox}
\def\dontdofcolorbox{\renewcommand\fcolorbox[4][]{##4}}
\makeatother

\newcommand{\textpf}[1]{\texttt{\color{black}\fcolorbox{lightgray}{lightgray}{#1}}}

\begin{document}

\begin{frame}[fragile]{Processes}
    \begin{itemize}
        \pause \item What's the difference between a process and a program?
        \pause \item A program is a file on the disk. It's just a sequence of bytes.
        \pause \item A process is a running program. That is, a process has an address space, file descriptors, etc.
    \end{itemize}
\end{frame}

\begin{frame}[fragile]{What does a shell do?}
    \begin{itemize}
        \pause \item When you run \textpf{./a.out}, what actually happens?
        \pause \item What system calls is the shell making to cause \textpf{a.out} to execute?
    \end{itemize}
\end{frame}

\begin{frame}[fragile]{\textpf{strace}}
    \begin{itemize}
        \pause \item \textpf{strace} is a tool that can show you the system call sequence for another process.
        \pause \item To use \textpf{strace} on \textpf{./a.out}, just run \textpf{strace ./a.out}.
        \pause \item Demo!
    \end{itemize}
\end{frame}

\begin{frame}[fragile]{\textpf{execve}}
    \begin{itemize}
        \pause \item \textpf{execve} is a syscall that asks the kernel to unmap this process's entire address space, then map in and run a new executable.
        \pause \item \textpf{execve} takes 3 arguments:
            \begin{itemize}
                \pause \item \textpf{char *path}: the path to the executable that we want to run.
                \pause \item \textpf{char *argv}: the \textpf{argv} that we want to pass. (May be \textpf{NULL})
                \pause \item \textpf{char *envp}: the \textpf{envp} that we want to pass. (May be \textpf{NULL})
            \end{itemize}
        \pause \item Demo!
    \end{itemize}
\end{frame}

\begin{frame}[fragile]{\textpf{execve} \textbf{replaces} the current process!}
    \begin{itemize}
        \pause \item Once you \textpf{execve}, the current process has been wiped out!
        \pause \item (Except a few things remain, like open file descriptors. Don't worry about this for now.)
    \end{itemize}
\end{frame}

\begin{frame}[fragile]{Making more processes}
    \begin{itemize}
        \pause \item If we \textit{only} had \textpf{execve}, we would have no way of creating new processes; we could only replace existing processes.
        \pause \item If you want to create a new process, you can use
            \begin{itemize}
                \pause \item \textpf{fork}, (easy to use, inefficient)
                \pause \item \textpf{vfork}, (easy to use, slightly less inefficient)
                \pause \item \textpf{clone}, (annoying to use, much more efficient)
                \pause \item \textpf{clone3}. (slightly less annoying to use, still efficient)
            \end{itemize}
    \end{itemize}
\end{frame}

\begin{frame}[fragile]{\textpf{fork}}
    \begin{itemize}
        \pause \item \textpf{fork} makes a duplicate of the existing process.
        \pause \item The new process is called a \textit{child process} of the original (\textit{parent}) process.
        \pause \item For the parent process, \textpf{fork} returns the PID of the child process.
        \pause \item For the child process, \textpf{fork} returns 0.
        \pause \item Demo!
    \end{itemize}
\end{frame}

\begin{frame}[fragile]{Returning to \textpf{strace}}
    Demo!
\end{frame}

\begin{frame}[fragile]{Activity}
    Fork bomb \textpf{plink}!
\end{frame}

\begin{frame}[fragile]{Field Trip!}
    We need to reboot \textpf{plink} :)
\end{frame}

\begin{frame}[fragile]{Race Conditions}
    \begin{itemize}
        \pause \item A \textit{race condition} occurs when a program behaves unpredictably due to simultaneously-executing elements.
        \pause \item Race conditions can cause lots of problems. Never write programs with race conditions.
    \end{itemize}
\end{frame}

\begin{frame}[fragile]{Activity}
    Using \textit{fork}, write a program with a race condition :)
\end{frame}

\begin{frame}[fragile]{Avoiding Race Conditions}
    \begin{itemize}
        \pause \item How do you avoid race conditions between processes?
            \begin{itemize}
                \pause \item \textpf{sleep}ing (not very effective)
                \pause \item \textpf{wait}ing (a little more effective)
                \pause \item \textpf{IPC} (can be very effective)
            \end{itemize}
        \pause \item We will discuss a few different IPC mechanisms in a few weeks.
    \end{itemize}
\end{frame}

\begin{frame}[fragile]{Activity}
    Fix the race condition in your program from earlier.
\end{frame}

\end{document}
