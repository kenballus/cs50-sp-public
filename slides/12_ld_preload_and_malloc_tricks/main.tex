\documentclass[hyphens,aspectratio=169,dvipsnames]{beamer}
\usepackage{graphicx}
\usepackage{xcolor}
\usepackage{amssymb}
\usepackage{amsmath}
\usepackage{mathtools}
\usepackage{textcomp}
\usepackage{moresize}
\usepackage{framed}
\usepackage{minted}
\usepackage{relsize}
\usepackage{emoji}
\usepackage{tikz}
\usepackage{comment}
\usetikzlibrary{shapes.geometric, arrows, positioning}
\usetheme{Berlin}
\usecolortheme[RGB={0,95,47}]{structure}
\beamertemplatenavigationsymbolsempty

\makeatletter
\AtBeginEnvironment{minted}{\dontdofcolorbox}
\def\dontdofcolorbox{\renewcommand\fcolorbox[4][]{##4}}
\makeatother

\newcommand{\textpf}[1]{\texttt{\color{black}\fcolorbox{lightgray}{lightgray}{#1}}}

\begin{document}

\begin{frame}[fragile]{Static Linking vs. Dynamic Linking}
    Recall that programs can be linked in 2 different ways:
    \begin{itemize}
        \pause \item Statically:
            \begin{itemize}
                \pause \item Calls to library functions are filled in when you run the compiler driver.
            \end{itemize}
        \pause \item Dynamically:
            \begin{itemize}
                \pause \item Calls to library functions are stubbed out when you run the compiler driver.
                \pause \item When you run the program, the dynamic linker (\textpf{/usr/lib/ld-linux-x86-64.so.2}) runs first, and maps in the needed libraries.
            \end{itemize}
    \end{itemize}
\end{frame}

\begin{frame}[fragile]{Static Linking vs. Dynamic Linking}
    Let's compare static and dynamic linking on the following criteria:
    \begin{itemize}
        \pause \item Disk usage. \pause \textbf{Dynamic wins!}
        \pause \item Memory usage. \pause \textbf{Could go either way (but usually dynamic wins)}
        \pause \item Patchability. \pause \textbf{Dynamic wins!} % Maybe think of a better word for this
        \pause \item Portability. \pause \textbf{Static wins!}
        \pause \item Program startup time. \pause \textbf{Static wins!}
    \end{itemize}
\end{frame}

\begin{frame}[fragile]{\textpf{LD\_PRELOAD}}
    \begin{itemize}
        \pause \item When the dynamic linker runs, you can request for it to override library functions with your own variants through the \textpf{LD\_PRELOAD} environment variable.
        \pause \item When this variable's value is a path to a shared library file, it loads that library first.
        \pause \item If there are name conflicts between \textpf{LD\_PRELOAD}ed libraries and system libraries, the \textpf{LD\_PRELOAD}ed libraries have priority!
    \end{itemize}
\end{frame}

\begin{frame}[fragile]{\textpf{1337\_puts.c}}
    \inputminted[fontsize=\tiny]{c}{1337_puts.c}
\end{frame}

\begin{frame}[fragile]{Building \textpf{1337\_puts.c}}
    \begin{minted}{bash}
gcc -shared 1337_puts.c -o 1337_puts.so
    \end{minted}
\end{frame}

\begin{frame}[fragile]{\textpf{LD\_PRELOAD}ing \textpf{1337\_puts.so}}
    Let's run \textpf{git --help}, overriding \textpf{puts} with the \textpf{1337} version :)
    \begin{minted}{bash}
LD_PRELOAD=./1337_puts.so git --help
    \end{minted}
\end{frame}

\begin{frame}[fragile]{Activity!}
    Write a funny version of a libc function, then \textpf{LD\_PRELOAD} it into something cool :)
\end{frame}

\begin{frame}[fragile]{Assignment 05: \textpf{malloc}}
    Let's discuss the assignment, and some ideas for making \textpf{malloc} implementations more efficient.
    
    Some ideas:
    \begin{itemize}
        \pause \item Store the previous chunk's size in each chunk to ease backward coalescing.
        \pause \item Store recently-freed chunks in either linked lists or arrays with counts.
        \pause \item When a chunk is free, you can repurpose its data section to store metadata.
    \end{itemize}
\end{frame}

\end{document}
