\documentclass[hyphens,aspectratio=169,dvipsnames]{beamer}
\usepackage{graphicx}
\usepackage{xcolor}
\usepackage{amssymb}
\usepackage{amsmath}
\usepackage{mathtools}
\usepackage{textcomp}
\usepackage{moresize}
\usepackage{framed}
\usepackage{minted}
\usepackage{relsize}
\usepackage{tikz}
\usepackage{comment}
\usetikzlibrary{shapes.geometric, arrows, positioning}
\usetheme{Berlin}
\usecolortheme[RGB={0,95,47}]{structure}
\beamertemplatenavigationsymbolsempty

\makeatletter
\AtBeginEnvironment{minted}{\dontdofcolorbox}
\def\dontdofcolorbox{\renewcommand\fcolorbox[4][]{##4}}
\makeatother

\newcommand{\textpf}[1]{\texttt{\color{black}\fcolorbox{lightgray}{lightgray}{#1}}}

\begin{document}

\begin{frame}[fragile]{Disclaimer}
    \begin{itemize}
        \pause \item This is a large, important lecture.
        \pause \item Please interrupt me as soon as something doesn't make sense.
        \pause \item I would legitimately rather not finish all of the material than have any of you not understand what's going on.
        \pause \item This lecture will require your complete attention :)
    \end{itemize}
\end{frame}

\begin{frame}[fragile]{Array Operations}
    In C, there are 2 fundamental array operations:
    \begin{enumerate}
        \pause \item Unary \textpf{*}, a.k.a. dereference
            \begin{itemize}
                \pause \item Accesses the first item in the array.
            \end{itemize}
        \pause \item Binary \textpf{+} with integer operand
            \begin{itemize}
                \pause \item Takes the tail of the array.
                \pause \item For example, adding 1 to an array in C is similar to slicing a list in Python with \textpf{[1:]}.
            \end{itemize}
    \end{enumerate}
\end{frame}

\begin{frame}[fragile]{Declaring a Local Array}
    \begin{itemize}
        \pause \item The following program declares a local array of 5 \textpf{unsigned int}s named \textpf{ary}, and returns its first element.
            \inputminted{c}{uninitialized_local_array.c}
        \pause \item Q: What does this program return?
        \pause \item A: \textbf{Indeterminate.}
    \end{itemize}
\end{frame}

\begin{frame}[fragile]{Declaring a Global Array}
    \begin{itemize}
        \pause \item The following program declares a global array of 5 \textpf{unsigned int}s named \textpf{ary}, and returns its first element.
            \inputminted{c}{uninitialized_global_array.c}
        \pause \item Q: What does this program return?
        \pause \item A: \textbf{0. All global variables are default-initialized to 0.}
        \pause \item This is unintuitive! You'll understand the reason for this later in the course.
    \end{itemize}
\end{frame}

\begin{frame}[fragile]{VLAs}
    \begin{itemize}
        \pause \item The length of an array must be a compile-time constant expression.
        \pause \item On some compilers, this isn't enforced, so the following would work:
            \inputminted{c}{vla.c}
        \pause \item Arrays where the size is not a compile-time constant are known as ``variable-length arrays" (VLAs).
        \pause \item They are error-prone and difficult to reason about, and were removed from the C standard in 2011. Do not use them!
        \pause \item From now on, always compile with \textpf{-Wvla} so the compiler will warn you when you accidentally use a VLA.
    \end{itemize}
\end{frame}

\begin{frame}[fragile]{Initializing an Array}
    \begin{itemize}
        \item \pause Just like \textpf{struct}s, arrays can be initialized with intializer lists:
        \inputminted{c}{initialized_local_array.c}
    \end{itemize}
\end{frame}

\begin{frame}[fragile]{Default-Initializing an Array}
    \begin{itemize}
        \pause \item Initializer lists are permitted to have fewer elements than the length of the array.
            \inputminted{c}{default_initialized_local_array.c}
        \pause \item Q: What is the second element in the array?
        \pause \item A: \textbf{When an array (global or local) is assigned to an initializer list with less elements than the array's length, the remaining elements are default-initialized to 0.}
        \pause \item This is really unintuitive!
    \end{itemize}
\end{frame}

\begin{frame}[fragile]{Default-Initializing an Array}
    \begin{itemize}
        \pause \item Q: What is the length of the array in this program?
            \inputminted{c}{implicit_length_local_array.c}
        \pause \item A: \textbf{1, thankfully :)}
    \end{itemize}
\end{frame}

\begin{frame}[fragile]{Array Indexing}
    \begin{itemize}
        \item \pause To index an array for element \textpf{n},
            \begin{itemize}
                \pause \item Add \textpf{n} to the array.
                \pause \item Dereference the result.
            \end{itemize}
        \pause \item Q: What does this return?
            \inputminted{c}{manual_array_index.c}
    \end{itemize}
\end{frame}

\begin{frame}[fragile]{Array Indexing}
    \begin{itemize}
        \pause \item Q: What does this return?
            \inputminted{c}{manual_array_index_2.c}
    \end{itemize}
\end{frame}

\begin{frame}[fragile]{The \textpf{[]} Operator}
    \begin{itemize}
        \pause \item The \textpf{[]} operator combines the \textpf{*} and \textpf{+} operators.
        \pause \item That is, the following is equivalent to the programs on the previous two slides:
            \inputminted{c}{regular_array_index.c}
        \pause \item Q: Copy this program down and build it. How does it behave when you index the array for
        \begin{itemize}
            \pause \item -1?
            \pause \item 10?
            \pause \item -10?
            \pause \item 999999999?
        \end{itemize}
    \end{itemize}
\end{frame}

\begin{frame}[fragile]{Some Fun Trivia}
    \begin{itemize}
        \pause \item Suppose that \textpf{a} is an array and \textpf{i} is an integer.
        \pause \item Then, \textpf{a[i]} and \textpf{*(a + i)} are \textbf{exactly equivalent}.
        \pause \item Knowing this, what does the following program do?
            \inputminted{c}{weird_array_index.c}
    \end{itemize}
\end{frame}

\begin{frame}[fragile]{Arrays and \textpf{sizeof}}
    \begin{itemize}
        \pause \item Recall that the \textpf{sizeof} operator returns the number of bytes used to represent its argument.
        \pause \item Knowing this, how would you generically take the length of a nonempty array?
        \pause \item \textbf{Given an array \textpf{ary}, its length is}
        \begin{minted}{c}
(sizeof ary) / (sizeof ary[0])
        \end{minted}
    \pause \item Q: Try writing a function that takes as input an array of \textpf{unsigned int} and returns its length. Does it behave the way you expect?
    \end{itemize}
\end{frame}

\begin{frame}[fragile]{Activity: Arrays and \textpf{sizeof}}
    \begin{itemize}
        \pause \item Define a \textpf{struct} with 4 fields:
            \begin{itemize}
                \item 1 \textpf{int},
                \item 2 \textpf{char}s,
                \item 1 \textpf{long}.
            \end{itemize}
        \pause \item Make an array containing 3 of your \textpf{struct}s.
        \pause \item What do you expect the \textpf{sizeof} your array is?
        \pause \item Take its size using \textpf{sizeof}. Is it what you expect?
        \pause \item Now, take the \textpf{sizeof} your array \textpf{+ 1}. Is it what you expect?
        \pause \item Try reordering the fields in your \textpf{struct}. How does it affect the \textpf{sizeof} computations? Try to make sense of this :)
    \end{itemize}
\end{frame}

\begin{frame}[fragile]{Intro to Pointers}
    \begin{itemize}
        \pause \item Recall the following program:
            \inputminted{c}{uninitialized_local_array.c}
        \pause \item Q: The type of \textpf{ary} is \textpf{unsigned int[5]}. What is the type of \textpf{ary + 1}?
        \pause \item A: \textbf{unsigned int *}
        \pause \item (Pronounced ``\textpf{unsigned int} star" or ``\textpf{unsigned int} pointer")
    \end{itemize}
\end{frame}

\begin{frame}[fragile]{Pointers}
    \begin{itemize}
        \pause \item The compiler is only smart enough to reason about arrays when their declaration is in scope.
        \pause \item When you add to (take the tail of) an array, or pass an array into a function, it \textbf{decays} into a ``pointer."
        \pause \item For the most part, a pointer is usable in exactly the same way as an array, except
            \begin{enumerate}
                \pause \item The \textpf{sizeof} a pointer is always constant. On \textpf{plink}, it's 8.
                \pause \item Pointers can be reassigned.
            \end{enumerate}
    \end{itemize}
\end{frame}

\begin{frame}[fragile]{What is a Pointer?}
    \begin{itemize}
        \pause \item You can think of a process's memory as a really big array of bytes.
        \pause \item A memory address, then, is just an index into that array.
        \pause \item \textbf{A pointer is just a memory address.}
        \pause \item On \textpf{plink}, pointers are 8 bytes. You can therefore cast them back and forth from \textpf{long} and they will still work.
    \end{itemize}
\end{frame}

\begin{frame}[fragile]{Pointer Operations}
    \begin{itemize}
        \pause \item \textpf{*}, \textpf{+}, and \textpf{[]} work exactly the same for pointers as they do for arrays.
    \end{itemize}
\end{frame}

\begin{frame}[fragile]{The \textpf{\&} (Address-of) Operator}
    \begin{itemize}
        \pause \item The unary \textpf{\&} operator takes the address of its operand.
        \pause \item For example, if you have an \textpf{int} \textpf{i}, you can obtain its address with \textpf{\&i}.
    \end{itemize}
\end{frame}

\begin{frame}[fragile]{What's the \textit{point}?}
    \begin{itemize}
        \pause \item What does this program return?
            \inputminted{c}{broken_add.c}
    \end{itemize}
\end{frame}

\begin{frame}[fragile]{What's the \textit{point}?}
    \begin{itemize}
        \pause \item What if we want \textpf{add\_one} to modify \textpf{i} in \textpf{main}?
        \pause \item Then, we need to make 3 changes.
    \end{itemize}
\end{frame}

\begin{frame}[fragile]{The Original Program}
    \inputminted{c}{broken_add.c}
\end{frame}

\begin{frame}[fragile]{Change \#1: Make \textpf{add\_one} take an \textpf{int *}.}
    \inputminted{c}{broken_add_p1.c}
\end{frame}

\begin{frame}[fragile]{Change \#2: Pass in \textpf{\&i} instead of \textpf{i}.}
    \inputminted{c}{broken_add_p2.c}
\end{frame}

\begin{frame}[fragile]{Change \#3: Dereference \textpf{p} before accessing it}
    \inputminted{c}{broken_add_p3.c}
\end{frame}

\begin{frame}[fragile]{Why does this work?}
    \inputminted{c}{broken_add_p3.c}
    \begin{itemize}
        \pause \item When \textpf{main} calls \textpf{add\_one}, it passes in the address of its local variable \textpf{i}.
        \pause \item When \textpf{add\_one} dereferences \textpf{p}, it accesses the same memory that stores \textpf{i}.
    \end{itemize}
\end{frame}

\end{document}
