\documentclass[hyphens,aspectratio=169,dvipsnames]{beamer}
\usepackage{graphicx}
\usepackage{xcolor}
\usepackage{amssymb}
\usepackage{amsmath}
\usepackage{mathtools}
\usepackage{textcomp}
\usepackage{moresize}
\usepackage{framed}
\usepackage{minted}
\usepackage{relsize}
\usepackage{tikz}
\usepackage{comment}
\usetikzlibrary{shapes.geometric, arrows, positioning}
\usetheme{Berlin}
\usecolortheme[RGB={0,95,47}]{structure}
\beamertemplatenavigationsymbolsempty

\makeatletter
\AtBeginEnvironment{minted}{\dontdofcolorbox}
\def\dontdofcolorbox{\renewcommand\fcolorbox[4][]{##4}}
\makeatother

\newcommand{\textpf}[1]{\texttt{\color{black}\fcolorbox{lightgray}{lightgray}{#1}}}

\begin{document}

\begin{frame}[fragile]{Multi-File C Programs}
    \begin{itemize}
        \pause \item Recall \textpf{my\_log2}, \textpf{my\_sub}, and \textpf{my\_abs} from last class.
        \pause \item Suppose that we want to use them in the following program:
    \end{itemize}
    \inputminted{c}{main.c}

    \pause \begin{center} \textbf{Task}: Copy down this program as-is into a file named \textpf{main.c} and attempt to compile it. \end{center}
\end{frame}


\begin{frame}[fragile]{Understanding the Errors}
    \scriptsize
    \begin{verbatim}
main.c: In function ‘main’:
main.c:2:12: error: implicit declaration of function ‘my_sub’ [-Wimplicit-function-declaration]
    2 |     return my_sub(my_log2(my_abs(18446744073709551615ul)));
      |            ^~~~~~
main.c:2:19: error: implicit declaration of function ‘my_log2’ [-Wimplicit-function-declaration]
    2 |     return my_sub(my_log2(my_abs(18446744073709551615ul)));
      |                   ^~~~~~~
main.c:2:27: error: implicit declaration of function ‘my_abs’ [-Wimplicit-function-declaration]
    2 |     return my_sub(my_log2(my_abs(18446744073709551615ul)));
      |
    \end{verbatim}
\end{frame}

\begin{frame}[fragile]{What is Implicit Function Declaration?}
    \pause \begin{center} The compiler considers a function  \textpf{f} to ``implicitly declared" when \textpf{f} is called before the compiler knows that \textpf{f} exists. \end{center}
\end{frame}

\begin{frame}[fragile]{Fixing Implicit Function Declaration}
    \begin{center} How do we fix this? \end{center}
    \begin{itemize}
        \pause \item Copy and paste the definitions of \textpf{my\_log2}, \textpf{my\_sub}, and \textpf{my\_abs} into \textpf{main.c}.
            \begin{itemize}
                \pause \item Doesn't scale well. We can't just cram everything into a single file!
            \end{itemize}
        \pause \item Define the functions in another file, then instruct the compiler to read both files.
    \end{itemize}
\end{frame}

\begin{frame}[fragile]{Activity \#1.0: Building a Multi-File Program}
    \begin{enumerate}
        \pause \item Take the definitions of \textpf{my\_log2}, \textpf{my\_sub}, and \textpf{my\_abs}, and place them in a new file named \textpf{my\_math.c}.
        \pause \item Attempt to compile both \textpf{my\_math.c} and \textpf{main.c} together by passing them both on the compiler driver command line.
        \begin{itemize}
            \pause \item Why doesn't this work?
            \pause \item Hint: The compiler considers a function  \textpf{f} to ``implicitly declared" when \textpf{f} is called before the compiler knows that \textpf{f} exists.
        \end{itemize}
    \end{enumerate}
\end{frame}

\begin{frame}[fragile]{Activity \#1.1: Telling the Compiler to Shut Up}
    \begin{itemize}
        \pause \item The compiler still fails with the same error, because under the hood, files are compiled one at a time, and then linked together.
        \pause \item Thus, when the compiler is building \textpf{main.c}, it still doesn't know about the functions in \textpf{my\_math.c}.
        \pause \item You can tell the compiler to stop worrying about this by passing the \textpf{-Wno-implicit-function-declaration} flag on the command line.
        \pause \item This works because the linker resolves function references, and the linker has access to the results of compiling both files.
    \end{itemize}
\end{frame}

\begin{frame}[fragile]{Activity \#1.2: Why \textpf{-Wno-implicit-function-declaration} is Bad}
    \begin{center} \textbf{NEVER USE \textpf{-Wno-implicit-function-declaration}!!!} \end{center}

    \begin{itemize}
        \pause \item The compiler doesn't know the argument counts and types for implicitly-declared functions.
        \pause \item The linker doesn't care about argument counts and types at all.
        \pause \item Try modifying \textpf{main.c} to call the functions with the wrong numbers of arguments. What happens?
    \end{itemize}
\end{frame}

\begin{frame}[fragile]{Activity \#1.3: Making the Compiler Happy}
    \begin{itemize}
        \pause \item To inform the compiler about your math functions in \textpf{main.c}, those functions must be declared.
        \pause \item A function declaration is just like a function definition, but no code block is provided.
        \pause \item For example, the declaration for \textpf{my\_log2} might look like this:
            \begin{minted}{c}
unsigned char my_log2(unsigned long x);
            \end{minted}
        \pause \item Modify \textpf{main.c} to add declarations for your math functions. This should resolve the compiler errors without using the evil flag.
    \end{itemize}
\end{frame}

\begin{frame}[fragile]{Activity \#1.4: Your First Header File}
    \begin{itemize}
        \pause \item This is still kind of ugly. Now your programs will be littered with function declarations.
        \pause \item To avoid this, function declarations are typically placed in their own files, called header files.
        \pause \item Make a header file, \textpf{my\_math.h}, and move the declarations from \textpf{main.c} into it.
        \pause \item \textpf{.h} files typically aren't compiled directly. Instead, you instruct the preprocessor to paste their contents into some \textpf{.c} file using the \textpf{\#include} preprocessor directive.
        \pause \item Modify \textpf{main.c} to \textpf{\#include} your header file in \textpf{main.c}.
    \end{itemize}
\end{frame}

\begin{frame}[fragile]{\textpf{struct}s}
    \begin{itemize}
        \pause \item In OOP, classes combine two primary things:
            \begin{enumerate}
                \pause \item Bundling together data types (aka, making record types)
                \pause \item Attaching functions to these data bundles.
            \end{enumerate}
        \pause \item In C, there are no objects and classes, but there \textit{are} \textpf{struct}s.
        \pause \item \textpf{struct}s are like classes, but they can't have methods.
    \end{itemize}
\end{frame}

\begin{frame}[fragile]{\textpf{struct} Definition}
    \textpf{struct}s are defined as follows:
    \begin{minted}{c}
struct example_struct {
    unsigned char first_field;
    int second_field;
    short third_field;
};
    \end{minted}
\end{frame}

\begin{frame}[fragile]{Populating a \textpf{struct}}
    To access in the fields in a \textpf{struct}, use the \textpf{.} operator.

    \pause For example, to make a \textpf{struct example\_struct}, and set its \textpf{second\_field} to \textpf{-1000},
    \begin{minted}{c}
struct example_struct s;
s.second_field = -1000;
    \end{minted}
\end{frame}

\begin{frame}[fragile]{Activity \#2: Fractions!}
    \begin{itemize}
        \pause \item Write a C file, \textpf{fractions.c}, that defines a \textpf{struct} that represents a fraction, along with implementations of fraction multiplication, division, addition, and subtraction.
        \pause \item Each of these functions should take as input two of your fraction \textpf{struct}s, and produce as output one fraction \textpf{struct}.
        \pause \item Don't worry about keeping fractions in simplest form.
        \pause \item When you're done, write a header file, \textpf{fractions.h}, that declares the functions defined in \textpf{fractions.c}. What else should go in that header file?
    \end{itemize}
\end{frame}

\end{document}
