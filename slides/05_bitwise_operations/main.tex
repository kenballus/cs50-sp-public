\documentclass[hyphens,aspectratio=169,dvipsnames]{beamer}
\usepackage{graphicx}
\usepackage{xcolor}
\usepackage{amssymb}
\usepackage{amsmath}
\usepackage{mathtools}
\usepackage{textcomp}
\usepackage{moresize}
\usepackage{framed}
\usepackage{minted}
\usepackage{relsize}
\usepackage{tikz}
\usepackage{comment}
\usetikzlibrary{shapes.geometric, arrows, positioning}
\usetheme{Berlin}
\usecolortheme[RGB={0,95,47}]{structure}
\beamertemplatenavigationsymbolsempty

\makeatletter
\AtBeginEnvironment{minted}{\dontdofcolorbox}
\def\dontdofcolorbox{\renewcommand\fcolorbox[4][]{##4}}
\makeatother

\newcommand{\textpf}[1]{\texttt{\color{black}\fcolorbox{lightgray}{lightgray}{#1}}}

\begin{document}

\begin{frame}[fragile]{Swapping Endianness}
    Write a function to swap the endianness of a \textpf{uint64\_t}, using only techniques we've discussed in class.
\end{frame}

\begin{frame}[fragile]{Terminology}
    \begin{itemize}
        \pause \item \textit{bit}: A system with exactly 2 states, typically called ``0" and ``1."
        \pause \item \textit{to sett a bit}: To set a bit's state to 1.
        \pause \item \textit{to unset (clear) a bit}: To set a bit's state to 0.
    \end{itemize}
\end{frame}

\begin{frame}[fragile]{Bitwise Operations}
    Here are a bunch of useful operators in C:
    \begin{itemize}
        \pause \item \textpf{<<}: Left Shift
        \pause \item \textpf{>>}: Right Shift
        \pause \item \textpf{\&}: Bitwise AND
        \pause \item \textpf{|}: Bitwise OR
        \pause \item \textpf{\^{}}: Bitwise XOR
        \pause \item \textpf{\textasciitilde}: Bitwise NOT
    \end{itemize}
\end{frame}

\begin{frame}[fragile]{\textpf{<<}: Left Shift}
    \begin{itemize}
        \pause \item Equivalently: multiplying by a power of 2.
        \pause \item How do you multiply a \textpf{uint64\_t} by 5 using only left shift and addition?
    \end{itemize}
\end{frame}

\begin{frame}[fragile]{\textpf{>>}: Right Shift}
    \begin{itemize}
        \pause \item Equivalently: integer dividing by a power of 2 (\textpf{//} from Python).
    \end{itemize}
\end{frame}

\begin{frame}[fragile]{\textpf{\&}: Bitwise AND}
    \begin{itemize}
        \pause \item Primary use: unsetting bits.
        \pause \item Equivalently: bitwise multiplication.
        \pause \item If you have a \textpf{uint64\_t}, how could you get its value mod 512 using only \textpf{\&}?
    \end{itemize}
\end{frame}

\begin{frame}[fragile]{\textpf{|}: Bitwise OR}
    \begin{itemize}
        \pause \item Primary use: setting bits.
        \pause \item Equivalently: addition without carry.
        \pause \item If you have a \textpf{uint64\_t}, how could you set its $n^\text{th}$ bit using bitwise OR?
    \end{itemize}
\end{frame}

\begin{frame}[fragile]{\textpf{\^{}}: Bitwise XOR}
    \begin{itemize}
        \pause \item Primary use: ``combining" values, flipping bits.
        \pause \item Equivalently: Bitwise \textpf{!=}.
        \pause \item If you have two \textpf{uint64\_t}s, how could swap them, using only \textpf{\^{}}, and no temporary variables?
    \end{itemize}
\end{frame}

\begin{frame}[fragile]{\textpf{\textasciitilde}: Bitwise NOT}
    \begin{itemize}
        \pause \item Inverts every bit in a number.
        \pause \item Equivalently: 2's complement minus 1.
    \end{itemize}
\end{frame}

\begin{frame}[fragile]{Why?}
    \begin{itemize}
        \pause \item \textbf{Performance!}
        \pause \item Most bitwise operations are single instructions for the CPU.
    \end{itemize}
\end{frame}

\begin{frame}[fragile]{Activity: Bit Tricks}
    \begin{enumerate}
        \pause \item Rewrite the endianness-swapping function from the beginning of class.
        \pause \item Write a function that takes a \textpf{uint64\_t} $i$ and a bit position $n$, and returns $i$ with the $n^\text{th}$ bit set.
        \pause \item Write a function that takes a \textpf{uint64\_t} $i$ and a bit position $n$, and returns $i$ with the $n^\text{th}$ bit unset.
        \pause \item Write a function that takes a \textpf{uint64\_t} $i$ and a bit position $n$, and returns the $n^\text{th}$ bit of $i$.
        \pause \item Write a function that takes a \textpf{uint64\_t}, and returns the index of its highest set bit. Does this seem familiar?
        \pause \item Write a function that multiplies its argument by 23, using only bitwise operations and addition.
    \end{enumerate}
\end{frame}

\begin{frame}[fragile]{Homework Discussion!}
    \begin{center}Homework Discussion!\end{center}
\end{frame}

\end{document}
