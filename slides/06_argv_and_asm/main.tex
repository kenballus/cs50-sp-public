\documentclass[hyphens,aspectratio=169,dvipsnames]{beamer}
\usepackage{graphicx}
\usepackage{xcolor}
\usepackage{amssymb}
\usepackage{amsmath}
\usepackage{mathtools}
\usepackage{textcomp}
\usepackage{moresize}
\usepackage{framed}
\usepackage{minted}
\usepackage{relsize}
\usepackage{tikz}
\usepackage{comment}
\usetikzlibrary{shapes.geometric, arrows, positioning}
\usetheme{Berlin}
\usecolortheme[RGB={0,95,47}]{structure}
\beamertemplatenavigationsymbolsempty

\makeatletter
\AtBeginEnvironment{minted}{\dontdofcolorbox}
\def\dontdofcolorbox{\renewcommand\fcolorbox[4][]{##4}}
\makeatother

\newcommand{\textpf}[1]{\texttt{\color{black}\fcolorbox{lightgray}{lightgray}{#1}}}

\begin{document}

\begin{frame}[fragile]{\textpf{argc} and \textpf{argv}}
    So far, we've used \textpf{main} functions with the following declaration:
    \begin{minted}{c}
int main(void);
    \end{minted}
\end{frame}

\begin{frame}[fragile]{\textpf{argc} and \textpf{argv}}
    It's also typical to use this declaration:
    \begin{minted}{c}
int main(int argc, char **argv);
    \end{minted}
\end{frame}

\begin{frame}[fragile]{\textpf{argc} and \textpf{argv}}
    It's also typical to use this declaration:
    \begin{minted}{c}
int main(int argc, char **argv);
    \end{minted}
\end{frame}

\begin{frame}[fragile]{\textpf{argc} and \textpf{argv}}
    \begin{itemize}
        \pause \item \textpf{argv} is an array of strings containing the \textit{command line arguments} passed to this program.
        \pause \item \textpf{argc} is its length.
    \end{itemize}
\end{frame}

\begin{frame}[fragile]{5-Minute-Activity: \textpf{cat}}
    Implement \textpf{cat}.
\end{frame}

\begin{frame}[fragile]{The CPU}
    \begin{itemize}
        \pause \item The CPU is the ``brain" of the computer.
        \pause \item CPUs have a fixed pool of operations that they can compute.
        \pause \item Each of these operations is known as an \textit{instruction}.
        \pause \item The CPU reads instructions encoded in ``machine code."
        \pause \item Machine code is annoying to write by hand, so humans typically write instructions in ``assembly language," which gets translated to machine code by an ``assembler."
    \end{itemize}
\end{frame}

\begin{frame}[fragile]{The Memory Hierarchy}
    Computers store data in a few different ways:
    \begin{itemize}
        \pause \item Disk: Large capacity (TB), slow access times (~1 ms), nonvolatile.
        \pause \item RAM: Medium capacity (GB), medium access times (~100 ns), volatile.
        \pause \item Registers: Tiny capacity (KB), fast access times ($<$1ns), volatile.
    \end{itemize}
\end{frame}

\begin{frame}[fragile]{RAM}
    \begin{itemize}
        \pause \item RAM is typically what we mean when we say ``memory."
        \pause \item Your mental model of the RAM should be that of a giant array of bytes.
        \pause \item Your CPU can provide the RAM with an address (a pointer), and the RAM provides back the byte stored at that address.
    \end{itemize}
\end{frame}

\begin{frame}[fragile]{Registers}
    \begin{itemize}
        \pause \item Registers are your CPU's ``scratch space."
        \pause \item Nearly all instructions operate on a register in some way.
        \pause \item Some registers are ``special-purpose," meaning that the CPU enforces that they must used for a particular purpose.
        \pause \item Other registers are ``general-purpose," meaning that the CPU allows you to use them for whatever you want.
        \pause \item On x86\_64, these are the general-purpose registers:
            \begin{itemize}
                \pause \item \textpf{rax}, \textpf{rbx}, \textpf{rcx}, \textpf{rdx}
                \pause \item \textpf{rdi}, \textpf{rsi}
                \pause \item \textpf{r8}, \textpf{r9}, ..., \textpf{r15}
            \end{itemize}
        \pause \item These are the special-purpose registers you need to remember:
            \begin{itemize}
                \pause \item \textpf{rip}
                \pause \item \textpf{rsp}, \textpf{rbp} (sort of in-between)
            \end{itemize}
    \end{itemize}
\end{frame}

\begin{frame}[fragile]{Register Widths}
    \begin{itemize}
        \pause \item All of these registers are 64 bits wide.
        \pause \item Sometimes, you just want to consider a portion of a register.
        \pause \item While masking is an option, you can also use a portion of a register directly.
        \pause \item For example, portions of the \textpf{rax} register can be accessed directly under the names:
            \begin{itemize}
                \pause \item \textpf{eax}: The low 32 bits of \textpf{rax}.
                \pause \item \textpf{ax}: The low 16 bits of \textpf{eax}.
                \pause \item \textpf{al}: The low 8 bits of \textpf{ax}.
            \end{itemize}
        \pause \item Most x86 registers support accessing smaller portions of large registers through the same naming scheme.
    \end{itemize}
\end{frame}

\begin{frame}[fragile]{FDE}
    From a high level, the CPU (\textbf{C}entral \textbf{P}rocessing \textbf{U}nit) does this, in a loop, forever:
    \begin{enumerate}
        \pause \item \textit{Fetch} an instruction from memory.
        \pause \item \textit{Decode} the instruction.
        \pause \item \textit{Execute} the instruction.
    \end{enumerate}
\end{frame}

\begin{frame}[fragile]{Where do we \textit{fetch}?}
    From what memory address does the CPU find the next instruction?
    \begin{itemize}
        \pause \item It depends!
        \pause \item On x86\_64, the special-purpose \textpf{rip} register stores the address of the next instruction to execute.
        \pause \item This register is typically called the ``instruction pointer" or ``program counter."
    \end{itemize}
\end{frame}

\begin{frame}[fragile]{What do we \textit{fetch}?}
    To fetch an instruction, how many bytes need to be read from \textpf{*rip}?
    \begin{itemize}
        \pause \item It depends!
        \pause \item On x86\_64, instructions are between 1 and 16 bytes long.
        \pause \item x86 CPUs have complicated internal logic to determine how much to read during an instruction fetch.
        \pause \item On ``RISC" CPUs (like ARM), instructions tend to be identically-sized.
        \pause \item On ``CISC" CPUs (like x86), instructions tend to be variably-sized.
    \end{itemize}
\end{frame}

\begin{frame}[fragile]{How to \textit{decode} a fetched instruction}
    Once an instruction has been fetched, it needs to be interpreted.
    A typical x86 instruction has 3 parts:
    \begin{enumerate}
        \pause \item \textit{Prefix bytes}: Bytes at the beginning of the instruction that modify its behavior.
        \pause \item \textit{Opcode}: A bit sequence that specifies the operation that the instruction signifies.
        \pause \item \textit{Operands}: Bit sequences that specify the operands to the instruction.
    \end{enumerate}
\end{frame}

\begin{frame}[fragile]{An Example}
    \[\textpf{48 05 00 00 00 0f}\]

    \pause \begin{center}...decodes to...\end{center}

    \pause \[\overbrace{\textpf{48 05}}^{\textpf{add rax}} \underbrace{\textpf{00 00 00 0f}}_{\textpf{0x0f000000}}\]
\end{frame}

\begin{frame}[fragile]{A First Assembly Program}
    \inputminted{asm}{return_25.s}
\end{frame}

\begin{frame}[fragile]{Building Assembly Programs}
    Just give it a \textpf{.s} file extension, and pass it to the compiler driver!

    \pause \begin{center} \textpf{gcc return\_25.s \&\& ./a.out; echo "\$?"} \end{center}
    \pause \begin{center} \textpf{25} \end{center}
\end{frame}

\begin{frame}[fragile]{Activities}
    Make an assembly program that
    \begin{enumerate}
        \item returns the sum of the values in \textpf{rdi}, \textpf{rsi}, and \textpf{rax},
        \item returns the $\log_2$ of the value in \textpf{rdi},
        \item returns the square root of the value in \textpf{rdi},
        \item returns the max of the values in all of the general-purpose registers.
    \end{enumerate}
    Expect to encounter strange problems, and ask me about them when they happen :)
\end{frame}

\end{document}
