\documentclass[hyphens,aspectratio=169,dvipsnames]{beamer}
\usepackage{graphicx}
\usepackage{xcolor}
\usepackage{amssymb}
\usepackage{amsmath}
\usepackage{mathtools}
\usepackage{textcomp}
\usepackage{moresize}
\usepackage{framed}
\usepackage{minted}
\usepackage{relsize}
\usepackage{tikz}
\usepackage{comment}
\usetikzlibrary{shapes.geometric, arrows, positioning}
\usetheme{Berlin}
\usecolortheme[RGB={0,95,47}]{structure}
\beamertemplatenavigationsymbolsempty

\makeatletter
\AtBeginEnvironment{minted}{\dontdofcolorbox}
\def\dontdofcolorbox{\renewcommand\fcolorbox[4][]{##4}}
\makeatother

\newcommand{\textpf}[1]{\texttt{\color{black}\fcolorbox{lightgray}{lightgray}{#1}}}

\begin{document}

\begin{frame}[fragile]{Virtual Addressing}
    \begin{itemize}
        \pause \item Demo!
        \pause \item Every process has its own address space.
        \pause \item This means that address $n$ in process $P_1$ does not refer to the same physical memory as address $n$ in process $P_2$.
        \pause \item The MMU (a component of the CPU) maps \textbf{virtual} addresses in a process's address space to \textbf{physical} addresses in the RAM.
    \end{itemize}
\end{frame}

\begin{frame}[fragile]{Memory Mappings}
    \begin{itemize}
        \pause \item Not every virtual address is actually backed by physical memory!
        \pause \item When you try to access memory at a virtual address that doesn't have any physical memory behind it, you get a \textpf{SIGSEGV}.
        \pause \item Virtual memory is mapped to physical memory in chunks (pages). On most systems, a page is 4096 (\textpf{0x1000}) bytes.
    \end{itemize}
\end{frame}

\begin{frame}[fragile]{Memory Permissions}
    \begin{itemize}
        \pause \item Memory has 3 permissions:
        \begin{itemize}
            \pause \item \textpf{R}: Whether this memory can be read from.
            \pause \item \textpf{W}: Whether this memory can be written to.
            \pause \item \textpf{X}: Whether this memory can be executed from.
        \end{itemize}
    \pause \item When you access memory in a way that violates its permissions, (e.g., when you attempt to write into read-only memory) you might get a \textpf{SIGSEGV}.
    \pause \item Q: Why do we do this?
    \end{itemize}
\end{frame}

\begin{frame}[fragile]{Viewing Memory Mappings in GDB}
    \begin{itemize}
        \pause \item When you're debugging a program in GDB, you can view its memory mappings with \textpf{info proc mappings}.
        \pause \item Demo!
    \end{itemize}
\end{frame}

\begin{frame}[fragile]{Viewing Memory Mappings Outside of GDB}
    \begin{itemize}
        \pause \item If you want to view the mappings of a program that's not running in GDB, do the following:
            \begin{enumerate}
                \pause \item Get its PID using \textpf{pidof}, \textpf{pgrep}, or \textpf{ps}.
                \pause \item \textpf{cat /proc/XXXXXX/maps}, where \textpf{XXXXXX} is the PID you just found.
            \end{enumerate}
        \pause \item Demo!
    \end{itemize}
\end{frame}

\begin{frame}[fragile]{Activity: Draw an Address Space}
    \begin{itemize}
        \pause \item Take one of your \textpf{malloc} activity programs from last class, and draw its address space.
        \pause \item How does its address space change during the course of execution?
    \end{itemize}
\end{frame}

\begin{frame}[fragile]{Getting More Memory}
    Last class, we saw how \textpf{malloc} can be used to acquire more memory.
    Under the hood, \textpf{malloc} is usually implemented with the following system calls:
    \begin{itemize}
        \pause \item \textpf{mmap}: Generic syscall for acquiring new memory mappings.
        \pause \item \textpf{brk}: Historical syscall for acquiring memory, not as flexible.
    \end{itemize}
\end{frame}

\begin{frame}[fragile]{Understanding \textpf{man 2 mmap}}
    The \textpf{mmap} arguments:
    \begin{itemize}
        \pause \item \textpf{void *addr}: The desired virtual address for the new mapping, or \textpf{NULL} if you don't care. Must be a multiple of the page size.
        \pause \item \textpf{size\_t len}: The desired number of bytes to map. Will be rounded up to a multiple of the page size.
        \pause \item \textpf{int prot}: The desired permissions for the newly-mapped memory. Almost always a bitwise-OR of the following constants:
            \begin{itemize}
                \pause \item \textpf{PROT\_READ}
                \pause \item \textpf{PROT\_WRITE}
                \pause \item \textpf{PROT\_EXEC}
            \end{itemize}
    \end{itemize}
\end{frame}

\begin{frame}[fragile]{Understanding \textpf{man 2 mmap}, continued}
    The \textpf{mmap} arguments, continued:
    \begin{itemize}
        \pause \item \textpf{int flags}: Provides more information about the requested memory. In this class, will probably be a bitwise-OR of the following constants:
            \begin{itemize}
                \pause \item \textpf{MAP\_PRIVATE}: States that this mapping is not shared between multiple processes.
                \pause \item \textpf{MAP\_ANONYMOUS}: States that this mapping is not backed by a file.
                \pause \item \textpf{MAP\_FIXED\_NOREPLACE}: States that this mapping should be placed at exactly the virtual address specified by \textpf{addr}, unless something is already mapped there, in which case the \textpf{mmap} should fail.
            \end{itemize}
    \end{itemize}
\end{frame}

\begin{frame}[fragile]{Understanding \textpf{man 2 mmap}, continued}
    The \textpf{mmap} arguments, continued:
    \begin{itemize}
        \pause \item \textpf{int fd}: Always pass \textpf{-1} for this argument.
        \pause \item \textpf{int offset}: Always pass \textpf{0} for this argument.
    \end{itemize}
\end{frame}

\begin{frame}[fragile]{Understanding \textpf{man 2 mmap}, continued}
    \textpf{mmap} returns either
    \begin{itemize}
        \pause \item the virtual address of the new mapping on success, or
        \pause \item a special constant \textpf{MAP\_FAILED} on failure.
    \end{itemize}
\end{frame}

\begin{frame}[fragile]{\textpf{munmap}}
    \begin{itemize}
        \pause \item When you're done with some \textpf{mmap}ed memory, \textpf{munmap} it!
        \pause \item Otherwise, you have a memory leak.
    \end{itemize}
\end{frame}

\begin{frame}[fragile]{An Example}
    \inputminted[fontsize=\tiny]{c}{mmap_example.c}
\end{frame}

\begin{frame}[fragile]{Activities}
    \begin{itemize}
        \pause \item Use \textpf{mmap} to map some read-only memory. Try writing into it. What happens?
        \pause \item Use \textpf{mmap} to map some write-only memory. Try reading from it. What happens?
        \pause \item Use \textpf{mmap} to map some readable and writable memory. Try executing from it (this may require assembly). What happens?
    \end{itemize}
\end{frame}

\end{document}
