\documentclass[hyphens,aspectratio=169,dvipsnames]{beamer}
\usepackage{graphicx}
\usepackage{xcolor}
\usepackage{amssymb}
\usepackage{amsmath}
\usepackage{mathtools}
\usepackage{textcomp}
\usepackage{moresize}
\usepackage{framed}
\usepackage{minted}
\usepackage{relsize}
\usepackage{tikz}
\usepackage{comment}
\usetikzlibrary{shapes.geometric, arrows, positioning}
\usetheme{Berlin}
\usecolortheme[RGB={0,95,47}]{structure}
\beamertemplatenavigationsymbolsempty

\makeatletter
\AtBeginEnvironment{minted}{\dontdofcolorbox}
\def\dontdofcolorbox{\renewcommand\fcolorbox[4][]{##4}}
\makeatother

\newcommand{\textpf}[1]{\texttt{\color{black}\fcolorbox{lightgray}{lightgray}{#1}}}

\begin{document}

\begin{frame}[fragile]{CS 94}
    \begin{center} Intro to Systems Programming \end{center}
\end{frame}

\begin{frame}[fragile]{About C}
    \pause C is
    \begin{itemize}
        \pause \item fast \pause (if you know what you're doing),
        \pause \item portable \pause (if you know what you're doing),
        \pause \item low-level \pause (useful for ``close-to-the-metal" tasks),
        \pause \item old \pause (nearly everything depends on C in some way),
        \pause \item unsafe, \pause (unless you \textbf{really} know what you're doing).
    \end{itemize}
\end{frame}

\begin{frame}[fragile]{A Simple C Program}
    Please copy down this C program into a file named \textpf{simple.c} on \textpf{plink}:
    \inputminted{c}{./simple.c}
\end{frame}

\begin{frame}[fragile]{Explanation of \textpf{simple.c}}
    \begin{itemize}
        \pause \item \textpf{int}
            \begin{itemize}
                \pause \item The return type of the \textpf{main} function.
            \end{itemize}
        \pause \item \textpf{main}
            \begin{itemize}
                \pause \item The name of the \textpf{main} function.
            \end{itemize}
        \pause \item \textpf{(void)}
            \begin{itemize}
                \pause \item Indicates that \textpf{main} takes no arguments.
                \pause \item You might expect that empty parentheses would indicate no arguments, but it actually indicates that the function accepts any number of arguments, and ignores them all.
            \end{itemize}
    \end{itemize}
\end{frame}

\begin{frame}[fragile]{Machine Code}
    \begin{itemize}
        \pause \item A computer's native language is known as ``machine code."
        \pause \item At the very bottom, your CPU's primary job is to interpret machine code.
    \end{itemize}
\end{frame}

\begin{frame}[fragile]{Interpreted Languages}
    \begin{itemize}
        \pause \item When you run a Python program, the Python interpreter translates your program into machine code at runtime.
        \pause \item This incurs runtime overhead.
    \end{itemize}
\end{frame}

\begin{frame}[fragile]{Compiled Languages}
    \begin{itemize}
        \pause \item In order to run a program written in a compiled language, you need to translate it into machine code ahead of time.
        \pause \item This process is known as ``compilation."
        \pause \item C is a compiled language.
    \end{itemize}
\end{frame}

\begin{frame}[fragile]{Building C Programs}
    There are 4 tools required to turn \textpf{simple.c} into an executable:
    \begin{enumerate}
        \pause \item The C preprocessor, which evaluates special ``directives," which I'll explain later.
        \pause \item The C compiler, which translates the output of the preprocessor into assembly.
        \pause \item The assembler, which translates assembly into machine code in the \textpf{.o} format.
        \pause \item The linker, which produces an executable or library from \textpf{.o} file(s).
    \end{enumerate}
\end{frame}

\begin{frame}[fragile]{The Compiler Driver}
    \begin{itemize}
        \pause \item A C compiler driver is a program that invokes the preprocessor, compiler, assembler, and linker for us, so that we can build C programs all in one step.
        \pause \item There are 2 compiler drivers installed on \textpf{plink}: \textpf{clang} and \textpf{gcc}.
        \pause \item Most of the time, these can be used interchangeably. Try both!
    \end{itemize}
\end{frame}

\begin{frame}[fragile]{Running the Compiler Driver}
    \begin{itemize}
        \pause \item To use \textpf{gcc} to build \textpf{simple.c}, run
        \begin{verbatim}
            gcc ./simple.c
        \end{verbatim}
        \pause \item You should now have an executable named ``\textpf{a.out}"
        \pause \item To execute \textpf{a.out}, run
        \begin{verbatim}
            ./a.out
        \end{verbatim}
        \pause \item To see the returned value from \textpf{main}, run
        \begin{verbatim}
            echo "$?"
        \end{verbatim}
    \end{itemize}
\end{frame}

\begin{frame}[fragile]{Integer Types in C}
    There are 4 basic integer types in C:
    \begin{itemize}
        \pause \item \textpf{char} (smallest),
        \pause \item \textpf{short} (small),
        \pause \item \textpf{int} (medium),
        \pause \item \textpf{long} (large).
    \end{itemize}
\end{frame}

\begin{frame}[fragile]{Integer Sizes}
    \begin{itemize}
        \pause \item C is designed for portability, so the language standard doesn't specify the number of bytes required to store each of the integer types.
        \pause \item For this reason, we need the \textpf{sizeof} operator.
        \pause \item The \textpf{sizeof} operator returns the number of bytes used to represent its operand.
    \end{itemize}
\end{frame}

\begin{frame}[fragile]{Activities}
    In this class, we'll intersperse activities with the lecture.
    In the activities, please don't use any concepts or library functions that haven't yet been introduced.
    This class is supposed to work from the ground up, and that means avoiding certain shortcuts for the time being.
\end{frame}

\begin{frame}[fragile]{Activity \#1}
    Modify \textpf{simple.c} to determine the sizes of the 4 basic integer types on \textpf{plink}.
\end{frame}

\begin{frame}[fragile]{Signedness}
    \begin{itemize}
        \pause \item All integers in C are either \textpf{signed} or \textpf{unsigned}.
        \pause \item Unsigned integers can only represent nonnegative values.
        \pause \item Signed integers can represent both negative and nonnegative values.
    \end{itemize}
\end{frame}

\begin{frame}[fragile]{Why would we ever want to use unsigned integers?}
    \begin{itemize}
        \pause \item When they better match the problem domain.
            \begin{itemize}
                \pause \item e.g., quantities, timestamps, etc.
            \end{itemize}
        \pause \item The positive range is larger!
            \begin{itemize}
                \pause \item The largest value that an \textpf{unsigned int} can store is greater than the largest value that a \textpf{signed int} can store.
            \end{itemize}
    \end{itemize}
\end{frame}

\begin{frame}[fragile]{Integer Ranges}
    \begin{itemize}
        \pause \item 1 bit can store one of $2^1 = 2$ distinct values: \textpf{0} or \textpf{1}.
        \pause \item 2 bits can store one of $2^2 = 4$ distinct bitstrings: \textpf{00}, \textpf{01}, \textpf{10}, or \textpf{11}.
        \pause \item Therefore, \textpf{char}, which is a 1-byte (8-bit) integer type, can store one of $2^8 = 256$ distinct bitstrings.
        \pause \item For \textpf{unsigned char}, we treat those bitstrings as representing $0$-$255$, inclusive.
        \pause \item For \textpf{signed char}, we treat those bitstrings as representing $-128$-$127$, inclusive.
    \end{itemize}
\end{frame}

\begin{frame}[fragile]{Activity \#2}
    Write a C function, \textpf{unsigned char my\_sub(unsigned char x, unsigned char y)}, \\that returns the difference between \textpf{x} and \textpf{y}.

    \begin{itemize}
        \pause \item Test your function on a wide range of inputs. Does it do what you expect?
    \end{itemize}
\end{frame}

\begin{frame}[fragile]{Activity \#3}
    \begin{center} Write a C function, \textpf{my\_abs(signed char x)}, that returns the absolute value of \textpf{x}. \end{center}

    \begin{itemize}
        \pause \item What should your function's return type should be?
            \begin{itemize}
                \pause \item \textpf{char}, \textpf{short}, \textpf{int}, or \textpf{long}?
                \pause \item \textpf{signed} or \textpf{unsigned}?
            \end{itemize}
        \pause \item Test your function on a wide range of inputs. Does it always work?
    \end{itemize}
\end{frame}

\begin{frame}[fragile]{Activity \#4}
    \begin{center} Write a C function, \textpf{my\_log2(unsigned long x)}, that returns the $\log_2$ of \textpf{x}. \end{center}

    \begin{itemize}
        \pause \item What should your function's return type should be?
            \begin{itemize}
                \pause \item \textpf{char}, \textpf{short}, \textpf{int}, or \textpf{long}?
                \pause \item \textpf{signed} or \textpf{unsigned}?
            \end{itemize}
        \pause \item When you're done, try implementing this function recursively (without helper functions).
    \end{itemize}
\end{frame}

\end{document}
