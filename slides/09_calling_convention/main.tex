\documentclass[hyphens,aspectratio=169,dvipsnames]{beamer}
\usepackage{graphicx}
\usepackage{xcolor}
\usepackage{amssymb}
\usepackage{amsmath}
\usepackage{mathtools}
\usepackage{textcomp}
\usepackage{moresize}
\usepackage{framed}
\usepackage{minted}
\usepackage{relsize}
\usepackage{tikz}
\usepackage{comment}
\usetikzlibrary{shapes.geometric, arrows, positioning}
\usetheme{Berlin}
\usecolortheme[RGB={0,95,47}]{structure}
\beamertemplatenavigationsymbolsempty

\makeatletter
\AtBeginEnvironment{minted}{\dontdofcolorbox}
\def\dontdofcolorbox{\renewcommand\fcolorbox[4][]{##4}}
\makeatother

\newcommand{\textpf}[1]{\texttt{\color{black}\fcolorbox{lightgray}{lightgray}{#1}}}

\begin{document}

\begin{frame}[fragile]{Making Your Assembly Compatible with C}
    \begin{itemize}
        \pause \item When you're writing assembly on your own, you can use every register however you like.
        % e.g., if you want to place your return value in rbx, this is fine.
        \pause \item When your assembly needs to be called by other people's C, then you need to conform to the ``calling convention."
    \end{itemize}
\end{frame}

\begin{frame}[fragile]{Argument Passing}
    On x86\_64 Linux, function arguments are passed in the following registers:
    \begin{enumerate}
        \pause \item \textpf{rdi}
        \pause \item \textpf{rsi}
        \pause \item \textpf{rdx}
        \pause \item \textpf{rcx}
        \pause \item \textpf{r8}
        \pause \item \textpf{r9}
    \end{enumerate}

    \pause \textbf{Note that this is distinct from syscall argument passing!!}
\end{frame}

\begin{frame}[fragile]{Return Values}
    On x86\_64 Linux, function return values are passed as follows:
    \begin{itemize}
        \pause \item If the value fits in 64 bits, it goes in \textpf{rax}.
        \pause \item Otherwise, if the value fits in 128 bits, it goes in \textpf{rdx:rax}.
        \pause \item Otherwise, don't worry about it :)
    \end{itemize}
\end{frame}

\begin{frame}[fragile]{Nonvolatile Registers}
    Some registers are assumed to not change across a function call. These are
    \begin{itemize}
        \pause \item \textpf{rbx},
        \pause \item \textpf{rbp},
        \pause \item \textpf{rsp},
        \pause \item \textpf{r12},
        \pause \item \textpf{r13},
        \pause \item \textpf{r14}, and
        \pause \item \textpf{r15}.
    \end{itemize}
    \begin{itemize}
        \pause \item If you want to use any of these registers in your functions, make sure to preserve them!
        \pause \item These registers are also called ``callee-preserved," because it is the responsibility of the callee (the function being called) to preserve them.
    \end{itemize}
\end{frame}

\begin{frame}[fragile]{Volatile Registers}
    \begin{itemize}
        \pause \item All other registers are allowed to change during a function call.
        \pause \item That means when you call a function, all of those registers could be changed arbitrarily by the function.
        \pause \item That means if you want to call a function, but you're using a volatile register for something, you need to preserve its value before calling the function.
        \pause \item These registers are also called ``caller-preserved," because it is the responsibility of the caller (the code that is making the \textpf{call}) to preserve them.
    \end{itemize}
\end{frame}

\begin{frame}[fragile]{Calling Assembly from C}
    To make an assembly function callable from C, do the following:
    \begin{itemize}
        \pause \item Ensure that your function conforms to the calling convention. That is, make sure that your function
            \begin{enumerate}
                \pause \item doesn't mess with stuff in the calling function's stack frame,
                \pause \item preserves the nonvolatile registers,
                \pause \item accepts arguments and returns in the expected registers.
            \end{enumerate}
        \pause \item Mark it as \textpf{.global}.
        \pause \item Declare it in a \textpf{.h} file.
    \end{itemize}
    \pause If you've done this, you can call your assembly function just like a normal C function!
\end{frame}

\begin{frame}[fragile]{Example: \textpf{add\_2.s}}
    \inputminted{asm}{./add_2.s}
\end{frame}

\begin{frame}[fragile]{Example: \textpf{add\_2.h}}
    \inputminted{c}{./add_2.h}
\end{frame}

\begin{frame}[fragile]{Example: \textpf{main.c}}
    \inputminted{c}{./main.c}
\end{frame}

\begin{frame}[fragile]{Activities!}
    Implement the following functions in assembly. Ensure that your code is callable from C.
    \begin{itemize}
        \pause \item \textpf{uint16\_t swapster(uint16\_t)}, which swaps the byte order of its argument.
            \begin{itemize}
                \pause \item e.g., \textpf{swapster(0xabcd)} returns \textpf{0xcdab}.
            \end{itemize}
        \pause \item \textpf{strcpy}, (feel free to call the \textpf{strlen} from libc)
        \pause \item \textpf{strcat}, (feel free to call the \textpf{strlen} from libc).
    \end{itemize}
\end{frame}

\end{document}
