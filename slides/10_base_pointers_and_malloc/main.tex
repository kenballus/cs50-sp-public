\documentclass[hyphens,aspectratio=169,dvipsnames]{beamer}
\usepackage{graphicx}
\usepackage{xcolor}
\usepackage{amssymb}
\usepackage{amsmath}
\usepackage{mathtools}
\usepackage{textcomp}
\usepackage{moresize}
\usepackage{framed}
\usepackage{minted}
\usepackage{relsize}
\usepackage{tikz}
\usepackage{comment}
\usetikzlibrary{shapes.geometric, arrows, positioning}
\usetheme{Berlin}
\usecolortheme[RGB={0,95,47}]{structure}
\beamertemplatenavigationsymbolsempty

\makeatletter
\AtBeginEnvironment{minted}{\dontdofcolorbox}
\def\dontdofcolorbox{\renewcommand\fcolorbox[4][]{##4}}
\makeatother

\newcommand{\textpf}[1]{\texttt{\color{black}\fcolorbox{lightgray}{lightgray}{#1}}}

\begin{document}

\begin{frame}[fragile]{An Example}
    \begin{itemize}
        \pause \item Let's look at 2 files: \textpf{frames.c} and \textpf{frames.s}.
        \pause \item Let's single-step this program by hand :)
    \end{itemize}
\end{frame}

\begin{frame}[fragile]{Stack Frames}
    \begin{itemize}
        \pause \item A function call's stack frame is the region of the stack that ``belongs to" that function call.
    \end{itemize}
\end{frame}

\begin{frame}[fragile]{Function Prologue}
    \begin{itemize}
        \pause \item Most functions that the compiler emits will begin with \textpf{push rbp}, \textpf{mov rbp, rsp}.
        \pause \item Q: What does this do?
        \pause \item A: This saves \textpf{rbp} onto the stack, then moves the current value of the stack pointer into \textpf{rbp}.
        \pause \item When \textpf{rbp} is used this way, it's acting as the ``base pointer," the pointer to the base of the stack frame.
    \end{itemize}
\end{frame}

\begin{frame}[fragile]{Stack Unwinding}
    \begin{itemize}
        \pause \item Q: Assuming that we are using \textpf{rbp} as a base pointer, what is stored at the base of each stack frame?
        \pause \item A: The saved base pointer from the previous stack frame!
        \pause \item Q: What's just above that on the stack?
        \pause \item A: The previous frame's return address!
        \pause \item When you use this to crawl through the stack and find all the previous return addresses, that's \textbf{stack unwinding}.
    \end{itemize}
\end{frame}

\begin{frame}[fragile]{Back to C!}
    :)
\end{frame}

\begin{frame}[fragile]{When Locals are Insufficient}
    \begin{itemize}
        \pause \item Recall this function signature UTF-8 assignment:
        \item {\tiny \textpf{uint64\_t parse\_utf8(uint8\_t *input, uint64\_t input\_len, uint32\_t *output);}}
        \pause \item Why did we need \textpf{output} to be passed into the function?
        \pause \item Why couldn't the function just return a pointer to a local array?
    \end{itemize}
\end{frame}

\begin{frame}[fragile]{\textpf{malloc}}
    \begin{itemize}
        \pause \item Let's read the \textpf{malloc} manual page.
        \pause \item Use \textpf{malloc} in these 3 scenarios:
            \begin{itemize}
                \pause \item When you need access to some memory, but you don't know how much at compile time.
                \pause \item When you need more memory than there is space on the stack (typically a few megabytes).
                \pause \item When you need to write a function that creates an object and returns its address.
                    \begin{itemize}
                        \pause \item This can often be avoided. How?
                    \end{itemize}
            \end{itemize}
    \end{itemize}
\end{frame}

\begin{frame}[fragile]{\textpf{free}}
    \begin{itemize}
        \pause \item Because \textpf{malloc}ed objects aren't scoped, you need to tell libc when you're done using them.
        \pause \item To do this, you use \textpf{free}.
        \pause \item Let's read the \textpf{free} manual page.
        \pause \item When you don't free memory that you \textpf{malloc}ed, that's a \textbf{memory leak}.
    \end{itemize}
\end{frame}

\begin{frame}[fragile]{Exercises}
    \begin{itemize}
        \pause \item Write \textpf{allocating\_strcat}, which is just like \textpf{strcat}, but doesn't modify its arguments.
        \pause \item Write a function that takes a \textpf{uint32\_t}, converts it to a string (in base 10), and returns the result.
        \pause \item Use \textpf{malloc} to modify your UTF-8 code to no longer need the \textpf{output} argument, and return a struct containing both the output and its size.
    \end{itemize}
\end{frame}

\end{document}
